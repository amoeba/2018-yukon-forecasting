\documentclass[]{article}
\usepackage{lmodern}
\usepackage{amssymb,amsmath}
\usepackage{ifxetex,ifluatex}
\usepackage{fixltx2e} % provides \textsubscript
\ifnum 0\ifxetex 1\fi\ifluatex 1\fi=0 % if pdftex
  \usepackage[T1]{fontenc}
  \usepackage[utf8]{inputenc}
\else % if luatex or xelatex
  \ifxetex
    \usepackage{mathspec}
  \else
    \usepackage{fontspec}
  \fi
  \defaultfontfeatures{Ligatures=TeX,Scale=MatchLowercase}
    \setmainfont[]{Source Sans Pro}
\fi
% use upquote if available, for straight quotes in verbatim environments
\IfFileExists{upquote.sty}{\usepackage{upquote}}{}
% use microtype if available
\IfFileExists{microtype.sty}{%
\usepackage{microtype}
\UseMicrotypeSet[protrusion]{basicmath} % disable protrusion for tt fonts
}{}
\usepackage[margin=1in]{geometry}
\usepackage{hyperref}
\PassOptionsToPackage{usenames,dvipsnames}{color} % color is loaded by hyperref
\hypersetup{unicode=true,
            colorlinks=true,
            linkcolor=Maroon,
            citecolor=Blue,
            urlcolor=blue,
            breaklinks=true}
\urlstyle{same}  % don't use monospace font for urls
\usepackage{longtable,booktabs}
\usepackage{graphicx,grffile}
\makeatletter
\def\maxwidth{\ifdim\Gin@nat@width>\linewidth\linewidth\else\Gin@nat@width\fi}
\def\maxheight{\ifdim\Gin@nat@height>\textheight\textheight\else\Gin@nat@height\fi}
\makeatother
% Scale images if necessary, so that they will not overflow the page
% margins by default, and it is still possible to overwrite the defaults
% using explicit options in \includegraphics[width, height, ...]{}
\setkeys{Gin}{width=\maxwidth,height=\maxheight,keepaspectratio}
\IfFileExists{parskip.sty}{%
\usepackage{parskip}
}{% else
\setlength{\parindent}{0pt}
\setlength{\parskip}{6pt plus 2pt minus 1pt}
}
\setlength{\emergencystretch}{3em}  % prevent overfull lines
\providecommand{\tightlist}{%
  \setlength{\itemsep}{0pt}\setlength{\parskip}{0pt}}
\setcounter{secnumdepth}{0}
% Redefines (sub)paragraphs to behave more like sections
\ifx\paragraph\undefined\else
\let\oldparagraph\paragraph
\renewcommand{\paragraph}[1]{\oldparagraph{#1}\mbox{}}
\fi
\ifx\subparagraph\undefined\else
\let\oldsubparagraph\subparagraph
\renewcommand{\subparagraph}[1]{\oldsubparagraph{#1}\mbox{}}
\fi

%%% Use protect on footnotes to avoid problems with footnotes in titles
\let\rmarkdownfootnote\footnote%
\def\footnote{\protect\rmarkdownfootnote}

%%% Change title format to be more compact
\usepackage{titling}

% Create subtitle command for use in maketitle
\newcommand{\subtitle}[1]{
  \posttitle{
    \begin{center}\large#1\end{center}
    }
}

\setlength{\droptitle}{-2em}
  \title{}
  \pretitle{\vspace{\droptitle}}
  \posttitle{}
  \author{}
  \preauthor{}\postauthor{}
  \date{}
  \predate{}\postdate{}


\begin{document}

\includegraphics{images/combined.png}

Monday, June 4, 2018

\begin{center}
  \huge{2018 Final Prospects and Forecast for Chinook Salmon Timing}
  
  \large{Lower Yukon River (Area Y-1)}
\end{center}

\hypertarget{run-timing-prospects-for-2018}{%
\subsection{Run Timing Prospects for
2018}\label{run-timing-prospects-for-2018}}

The 2018 timing for Yukon River Chinook is expected to be slightly
earlier than average. The first significant pulse (15\% point) of
Chinook in the lower river is expected on June 9, which is earlier than
the long-term average of June 13 (1961 -- 2017). The half-way point
(50\%) of the run is expected on June 17, which is also earlier than the
long-term average of June 21 (1961 -- 2017). The April mean air
temperature at Nome this year of -3.2°C was warmer than the long-term
average of -6.8°C (1961 -- 2017). The May mean sea surface temperature
as modeled just off the Yukon River delta (63.18°N 165.58°W) of -0.2°C
was warmer than the long-term average of -0.4°C (1961 -- 2017). In
keeping with the early spring conditions in northwestern Alaska, the
Spring (March 19 -- May 31) sea ice coverage was 11.85\%, which is very
far outside the experience of the historical database (1970 -- 2017),
which has a long-term average of 56.27\%, as measured in the waters
between the Yukon River delta and St.~Lawrence Island (Shpanberg
Strait).

\hypertarget{run-timing-forecast-by-date-2018}{%
\subsection{Run Timing Forecast by Date
2018}\label{run-timing-forecast-by-date-2018}}

Based on the relationship between the dates of the three percentiles
(15\%, 25\%, and 50\%) and the sea ice coverage and the air and sea
surface temperatures, the predicted dates of the three percentiles are
June 9 (15\%), June 11 (25\%), and June 17 (50\%) in District Y-1 of the
Lower Yukon River.

\begin{longtable}[]{@{}ll@{}}
\toprule
Percentile (\%) & Predicted Date\tabularnewline
\midrule
\endhead
15 & June 9\tabularnewline
25 & June 11\tabularnewline
50 & June 17\tabularnewline
\bottomrule
\end{longtable}

See
\href{http://www.aoos.org/2018-run-timing-outlook-and-forecast-summary-chinook-salmon-yukon-river-delta/}{the
project website} for comparisons of the estimated daily percentages of
the run to results of the 2018 Lower Yukon Test Fishery and past years'
test fisheries. If the current timing model doesn't match reality, the
forecast dates will be updated and posted on the web site.

\hypertarget{credits}{%
\subsection{Credits}\label{credits}}

Prepared and reviewed by Bryce Mecum
\href{mailto:brycemecum@gmail.com}{\nolinkurl{brycemecum@gmail.com}},
Phil Mundy
\href{mailto:proymundy@gmail.com}{\nolinkurl{proymundy@gmail.com}}, and
Jordan Watson
\href{mailto:jordan.watson@noaa.gov}{\nolinkurl{jordan.watson@noaa.gov}}.
Data management and web page support by Will Koeppen
\href{mailto:will@axiomalaska.com}{\nolinkurl{will@axiomalaska.com}}.
Web support also provided by Holly Kent
\href{mailto:\%20kent@aoos.org}{kent@aoos.org}. Yukon Chinook in-season
salmon data and management agency coordination by Fred West
\href{mailto:fred.west@alaska.gov}{\nolinkurl{fred.west@alaska.gov}},
Sean Larson
\href{mailto:sean.larson@alaska.gov}{\nolinkurl{sean.larson@alaska.gov}},
and Holly Carroll
\href{mailto:holly.carroll@alaska.gov}{\nolinkurl{holly.carroll@alaska.gov}}.
Financial and material supports were provided by the Alaska Ocean
Observing System, NOAA National Marine Fisheries Service, and the Alaska
Department of Fish and Game.


\end{document}
